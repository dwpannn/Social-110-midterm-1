\documentclass[12pt]{article}

\usepackage[utf8]{inputenc}
\usepackage{latexsym,amsfonts,amssymb,amsthm,amsmath}
\usepackage{setspace}
\doublespacing
\newcommand\tab[1][1cm]{\hspace*{#1}}


\setlength{\parindent}{0in}
\setlength{\oddsidemargin}{0in}
\setlength{\textwidth}{6.5in}
\setlength{\textheight}{8.8in}
\setlength{\topmargin}{0in}
\setlength{\headheight}{18pt}



\title{SOCIOL 110 Take-Home Midterm Exam #1}
\author{\textbf{David Pan} \\ SID: 3034151358}

\begin{document}

\maketitle

\vspace{0.5in}


\pagebreak
\subsection*{Question \#1 (20 points)}
\textbf{What the properties of the higher education space in the U.S. would have to be if we were to see the emergence of such specialist organizations? }\\


\textbf{\emph{Response: }} \\
\tab I believe such higher education space would champion and revere the critical role of multidisciplinary fields, which has been evidently abandoned by most of the  ``generalist colleges and universities" mentioned in the question. Namely, such space would recognize the fact that multidisciplinary fields contribute to the variance and the depth of traditionally popular fields like STEM. \\
\tab And this recognition is crucial to such space because it guarantees the faith of pursuing such kind of study from a epistemological perspective for incoming students. For instance, a student were told there would be no tomorrow because of the sudden invasion of Canada, he would have not done any assignments. He simply does not see a future with any utility rewards for his completion of assignments. This logic applies to incoming students who wants to explore multidisciplinary fields as well: if everyone around them, including the higher institution itself, keeps judging a student's value by how many monetary value he could produce, rather than how much he could contribute to the variance and depth of those in academia who do receive rewarding utilities for their studies, these students easily do not see the future with their contribution in it. This trajectory of career path and personal worth naturally discourages studying multidisciplinary fields in a ``generalist colleges and universities" societal setting.  \\
\tab To put above argument in a quantitative context, this hypothetical higher education space would religiously follow the The Breeder's Equation as a derivation for the theoretical development of the academia and eventually the human knowledge of society itself. The Breeder's Equation states that
\begin{equation}
\Delta Z = h^2 \times S
\end{equation}
where $Z$ is the mean of change of the evolutionary traits, $h^2$ is the proportion of trait variation statistically attributed to additive genetic effects, and $S$ is the selection differential, which could be both natural and artificial(``Kelly, J. K. (2011) The breeder's equation. Nature Education Knowledge 4(5):5"). And this equation matters to this hypothetical education place because it narrows down the progress of making evolutionary changes in human traits down to the improvement of artificial selection. In other words, since $S$ is a differential between ``trait values and fitness", it could be increased by reinforcing the association between the two values. And this increase could be done on an institutional level. For instance, practices like artificially lowering the prestige of selected STEM fields by spreading the ideal that multidisciplinary fields produce equal utility increase the fitness of those who pursue the study of multidisciplinary fields, thus increasing $S$ and $\Delta Z$. \\
\tab Of course, the recognition proposed and proved above could not be a standalone property. Things like implementing general liberal art education as introductory courses could potentially help with building the recognition as well. The recognition itself, however, separates this ideal higher education place from contemporary generalist colleges and universities that are wildly flooded by popular STEM majors. 

%\vspace{2in} %Leave space for comments!

\pagebreak
\subsection*{Question \#2 (20 points)}
\textbf{Explain whether the data supports the conclusion that Museums 1 and 2 are externally controlled by their dependence upon funders in terms of types of exhibits put on display, and why you think so.}\\
\textbf{\emph{Response (part a): }} \\
\tab The data supports the conclusion that Museums 1 and 2 are externally controlled by their dependence upon funders in terms of types of exhibits put on displays. \\
\tab The museum is an organization and the selection of display is controlled by the curator. In an ideal situation, where the museum is not controlled by the external dependence, the preference of the curator would play a critical role in the actual display. This, however, is not the case for museum 1 or museum 2. Indeed, the percentage for curator's preference in the display is $1 \div 10 = 10\%$ for both museums, a far cry from the percentage yielded from the ideal assumption. In other words, museum 1 and museum 2 are externally controlled organizations because the agency of internal controlling unit is not reflected on the eventual performance of both organizations. \\
\tab Furthermore, the breakdown of the funding of the organization almost proves a linear relationship with its performance. Namely, for museum 1 corporation funding consists of $70M \div 100M = 70\%$ of the total budget. As a result, the selection of the artwork linearly matches the preference of funding institution: $7 \div (7+2+1) = 70\%$. Similarly, the percentages of ``Accessible" and ``Scholarly" also follow the 










\end{document}
