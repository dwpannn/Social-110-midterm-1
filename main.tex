\documentclass[12pt]{article}

\usepackage[utf8]{inputenc}
\usepackage{latexsym,amsfonts,amssymb,amsthm,amsmath}
\usepackage{setspace}
\doublespacing
\newcommand\tab[1][1cm]{\hspace*{#1}}


\setlength{\parindent}{0in}
\setlength{\oddsidemargin}{0in}
\setlength{\textwidth}{6.5in}
\setlength{\textheight}{8.8in}
\setlength{\topmargin}{0in}
\setlength{\headheight}{18pt}



\title{SOCIOL 110 Take-Home Midterm Exam #1}
\author{\textbf{David Pan} \\ SID: 3034151358}

\begin{document}

\maketitle

\vspace{0.5in}


\pagebreak
\subsection*{Question \#1 (20 points)}
\textbf{What the properties of the higher education space in the U.S. would have to be if we were to see the emergence of such specialist organizations? }\\


\textbf{\emph{Response: }} \\
\tab I believe such higher education space would champion and revere the critical role of multidisciplinary fields, which has been evidently abandoned by most of the  ``generalist colleges and universities" mentioned in the question. Namely, such space would recognize the fact that multidisciplinary fields contribute to the variance and the depth of traditionally popular fields like STEM. \\
\tab And this recognition is crucial to such space because it guarantees the faith of pursuing such kind of study from a epistemological perspective for incoming students. For instance, a student were told there would be no tomorrow because of the sudden invasion of Canada, he would have not done any assignments. He simply does not see a future with any utility rewards for his completion of assignments. This logic applies to incoming students who wants to explore multidisciplinary fields as well: if everyone around them, including the higher institution itself, keeps judging a student's value by how many monetary value he could produce, rather than how much he could contribute to the variance and depth of those in academia who do receive rewarding utilities for their studies, these students easily do not see the future with their contribution in it. This trajectory of career path and personal worth naturally discourages studying multidisciplinary fields in a ``generalist colleges and universities" societal setting.  \\
\tab To put above argument in a quantitative context, this hypothetical higher education space would religiously follow the The Breeder's Equation as a derivation for the theoretical development of the academia and eventually the human knowledge of society itself. The Breeder's Equation states that
\begin{equation}
\Delta Z = h^2 \times S
\end{equation}
where $Z$ is the mean of change of the evolutionary traits, $h^2$ is the proportion of trait variation statistically attributed to additive genetic effects, and $S$ is the selection differential, which could be both natural and artificial(``Kelly, J. K. (2011) The breeder's equation. Nature Education Knowledge 4(5):5"). And this equation matters to this hypothetical education place because it narrows down the progress of making evolutionary changes in human traits down to the improvement of artificial selection. In other words, since $S$ is a differential between ``trait values and fitness", it could be increased by reinforcing the association between the two values. And this increase could be done on an institutional level. For instance, practices like artificially lowering the prestige of selected STEM fields by spreading the ideal that multidisciplinary fields produce equal utility increase the fitness of those who pursue the study of multidisciplinary fields, thus increasing $S$ and $\Delta Z$. \\
\tab Of course, the recognition proposed and proved above could not be a standalone property. Things like implementing general liberal art education as introductory courses could potentially help with building the recognition as well. The recognition itself, however, separates this ideal higher education place from contemporary generalist colleges and universities that are wildly flooded by popular STEM majors. 



\pagebreak
\subsection*{Question \#2 (20 points)}
\textbf{Explain whether the data supports the conclusion that Museums 1 and 2 are externally controlled by their dependence upon funders in terms of types of exhibits put on display, and why you think so.}\\
\textbf{\emph{Response (part a): }} \\
\tab The data supports the conclusion that Museums 1 and 2 are externally controlled by their dependence upon funders in terms of types of exhibits put on displays. \\
\tab The museum is an organization and the selection of display is controlled by the curator. In an ideal situation, where the museum is not controlled by the external dependence, the preference of the curator would play a critical role in the actual display. This, however, is not the case for museum 1 or museum 2. Indeed, the percentage for curator's preference in the display is $1 \div 10 = 10\%$ for both museums, a far cry from the percentage yielded from the ideal assumption. In other words, museum 1 and museum 2 are externally controlled organizations because the agency of internal controlling unit is not reflected on the eventual performance of both organizations. \\
\tab Furthermore, the breakdown of the funding of the organization almost proves a linear relationship with its performance. Namely, for museum 1 corporation funding consists of $70M \div 100M = 70\%$ of the total budget. As a result, the selection of the artwork linearly matches the preference of funding institution: $7 \div (7+2+1) = 70\%$. Similarly, the percentages of ``Accessible" and ``Scholarly" also follow the linearly proportional relationship with the preference of those funding institutions. And this relationship could also be found in museum 2, which resembles a symmetric interpretation of the funding of the organization: the funders' preference will be translated to corresponding selection of the artwork. Hence, museum 1 and museum 2 are not only externally controlled as organizations, they are specifically controlled by their dependence upon their funding institutions.\\
\pagebreak \\
\textbf{How is it that Museum 3 put on so many scholarly exhibits? Whose interests are behind the large number of scholarly exhibits, and why have those interests prevailed for Museum 3 and not Museums 1 or 2?}\\
\textbf{\emph{Response (part b): }} \\
\tab Museum 3 puts on so many scholarly exhibits because the majority of its funding is not associates with a clear preference. Namely, $70M \div (70M + 15M \times 2) = 70\%$ of the funding of this organization exerts \textbf{\emph{unspecified}} preference on the curating of the artwork. In other words, when the majority of the external influence gives no clear direction to the museum, the internal unit of the organization start to function. The overall curating preference starts to shift towards internal preference. This shift is powerful: it reveals how an organization responds to the change of its funding entity: government and corporations are both \textbf{\emph{institutions}}, which embrace uniformed goals. Small donations, on the other hand, do not resemble the same goal. \\
\tab Both the small donations' interests and the curator's interests are behind the large number of exhibits. Since the preference of small donations is \textbf{\emph{unspecified}}, the curator instill his own preference over the selection of the exhibits to optimize the satisfaction of the funding entities.  \\
\tab Above interests have not prevailed because for Museum 1 and 2 the funding entities are mostly institutions. Because of the nature of the institution, they share a common goal, which translation to a uniformed preference of the selection of the artwork. For Museum 3, this same proportion of institution is absent, leaving the majority of the donating body no strong preference. Hence, the curator has to step in and use his conscientiousness to curate the final selection.

\pagebreak
\subsection*{Question \#3 (20 points)}
\textbf{Provide an \emph{institutionalist} explanation for the structure of the beer industry.} \\
\textbf{\emph{Response (part a): }} \\
\tab Institutionalist explanation traces back to the role of institutions in shaping economic behaviors. In this big beer versus craft beer context, these economic behaviors translate to different volume of market share and the size of the brewery. From an institutionalist perspective, the split of market share is the result of the interactions among different breweries (institutions). Ans this interaction is muti-faceted because each institution has its own goal and institutional rationality. The differences are further augmented by the nature of the ownership. Namely, the big beer companies are mostly publicly traded companies which has to shape its institutional rationality from the economic behaviors of the shareholders. On the other hand, the privately or semi-privately owned craft beer breweries set their institutional logic from a different pipeline because the source of funding is fundamentally different from those of the big beer companies. From question 2 we know that an institution would behave differently under different funding pattern with different preferences. This lemma is still true in this question's context because one could easily argue shareholders and private business owners value different traits in a beer manufacturing process. From the quote one could tell craft beer breweries champion the variance of their values and their independence that derives from their variance, saying ``Our fans tell us all the time how important it is to them to know that Stone is steadfastly independent". While this variance is not a guarantee for better quality, the seal shows BA's support for such variance. Since the value varies between the big beer corporations and craft beer breweries, they must each have their own institutional goal, which gives them both rationales to act in their own way to produce complex economic behaviors. This cycle eventually converges economically in terms of the targeted audience and creates a hierarchy in the structure of beer manufacturing. \\
\pagebreak \\
\textbf{Explain what evidence there is from the article that customers of craft brewing organizations “are not getting what they need” (to paraphrase Dobbin/Kim/Kalev) out of the Brewers Assocation’s push for the seal.} \\
\textbf{\emph{Response (part b): }} \\
\tab While the seal mentioned in the article does make the ownership of a brewery clear to customers, it does not offer information that is objectively associated with the quality of the beer. As argued in part 1 of this question, the seal gives information about the variance of value and consequently the independence status of breweries, it is not necessarily correlated to the objective freshness and taste of the beer. One instance proposed in the article, the manufactured date would provide more information about the quality of the beer for the customers, as the projected economic behaviors that is linked with the seal gives the customers far more complex information when they go to their local gas stations and get their six-packs.






\end{document}
